% Chapter Template

\chapter{Conclusion} % Main chapter title

%\label{ChapterX} % Change X to a consecutive number; for referencing this chapter elsewhere, use \ref{ChapterX}
%\section{}
In this thesis KLFDA is used to reduce dimension of hierarchical gaussian descriptors, gradient descent method is used to learn a Mahanalobis distance matrix on lower-dimensional space. By comparison we can find the proposed metric has better performance than NFST and XQDA on VIPeR and CUHK1 datasets, but XQDA and NFST outperforms the proposed metric learning on Prid\_2011 and Prid\_450s, and the proposed metric learning has better rank 1 score than NFST and its performance is only second to XQDA on GRID dataset. 
\section{Contributions}
There are three contributions in this thesis: (1) The metric learning on dimension-reduced hierarchical gaussian descriptor by KLFDA has been studied. The gradient descent method optimizes the Mahanalobis distance matrix on the lower-dimensional space. It has been demonstrated that extra improvements can be achieved in this lower-dimensional space. (2) The influence of background and foreground segmentation on different descriptors have been fully studied. The results are that foreground segmentation improves performance of color based descriptors but decreases performance of texture based descriptors, which is caused by imperfect segmentation on single image based foreground segmentation. (3) Some variants of hierarchical gaussian descriptor have been tested, LBP and superpixel segmentation are combined with hierarchical gaussian descriptor but those variants have worse performance than original hierarchical gaussian descriptor.
\section{Future work}
\subsection{Improve Hierarchical Gaussian descriptor}
It has been demonstrated that in the basic pixel feature $\bm{f}_i$, the color components are more important than $y$ coordinate and gradient components. LBP has been turned to be a worse choice for texture representation in $\bm{f}_i$. Therefore, one strategy to improve hierarchical gaussian descriptor is to find a better texture representation to replace gradient components in basic pixel feature. 
\subsection{Influence of video-based foreground segmentation}
Though the single-image-based foreground and background segmentation's influence on different descriptors has been studied, extra effort is needed to study sequence or video-based segmentation's influence on different descriptors. Video-based foreground often has better results. If the background can be well modelled by a video or a sequence of images so that less extra textural noise is created, the segmentation might improve hierarchical gaussian descriptor's performance.
\subsection{Computational cost of gradient descent method}
The last one is to reduce the computational cost of gradient descent method. It takes about average 15 iterations when training the Mahanalobis distance matrix. The training time for larger dataset like CUHK dataset takes up to one hour. Therefore, other variants of gradient descent method like stochastic gradient method, conjugate gradient method may be tested for lower computational cost.


%----------------------------------------------------------------------------------------
%	SECTION 1
%----------------------------------------------------------------------------------------

%\section{Main Section 1}
%
%Lorem ipsum dolor sit amet, consectetur adipiscing elit. Aliquam ultricies lacinia euismod. Nam tempus risus in dolor rhoncus in interdum enim tincidunt. Donec vel nunc neque. In condimentum ullamcorper quam non consequat. Fusce sagittis tempor feugiat. Fusce magna erat, molestie eu convallis ut, tempus sed arcu. Quisque molestie, ante a tincidunt ullamcorper, sapien enim dignissim lacus, in semper nibh erat lobortis purus. Integer dapibus ligula ac risus convallis pellentesque.
%
%%-----------------------------------
%%	SUBSECTION 1
%%-----------------------------------
%\subsection{Subsection 1}
%
%Nunc posuere quam at lectus tristique eu ultrices augue venenatis. Vestibulum ante ipsum primis in faucibus orci luctus et ultrices posuere cubilia Curae; Aliquam erat volutpat. Vivamus sodales tortor eget quam adipiscing in vulputate ante ullamcorper. Sed eros ante, lacinia et sollicitudin et, aliquam sit amet augue. In hac habitasse platea dictumst.
%
%%-----------------------------------
%%	SUBSECTION 2
%%-----------------------------------
%
%\subsection{Subsection 2}
%Morbi rutrum odio eget arcu adipiscing sodales. Aenean et purus a est pulvinar pellentesque. Cras in elit neque, quis varius elit. Phasellus fringilla, nibh eu tempus venenatis, dolor elit posuere quam, quis adipiscing urna leo nec orci. Sed nec nulla auctor odio aliquet consequat. Ut nec nulla in ante ullamcorper aliquam at sed dolor. Phasellus fermentum magna in augue gravida cursus. Cras sed pretium lorem. Pellentesque eget ornare odio. Proin accumsan, massa viverra cursus pharetra, ipsum nisi lobortis velit, a malesuada dolor lorem eu neque.
%
%%----------------------------------------------------------------------------------------
%%	SECTION 2
%%----------------------------------------------------------------------------------------
%
%\section{Main Section 2}
%
%Sed ullamcorper quam eu nisl interdum at interdum enim egestas. Aliquam placerat justo sed lectus lobortis ut porta nisl porttitor. Vestibulum mi dolor, lacinia molestie gravida at, tempus vitae ligula. Donec eget quam sapien, in viverra eros. Donec pellentesque justo a massa fringilla non vestibulum metus vestibulum. Vestibulum in orci quis felis tempor lacinia. Vivamus ornare ultrices facilisis. Ut hendrerit volutpat vulputate. Morbi condimentum venenatis augue, id porta ipsum vulputate in. Curabitur luctus tempus justo. Vestibulum risus lectus, adipiscing nec condimentum quis, condimentum nec nisl. Aliquam dictum sagittis velit sed iaculis. Morbi tristique augue sit amet nulla pulvinar id facilisis ligula mollis. Nam elit libero, tincidunt ut aliquam at, molestie in quam. Aenean rhoncus vehicula hendrerit.